\chapter{Symmetric rank-k update: $ C := A^TA + C $ \\
	\large Team: Laurence Zhang, Yvonne Huang, Cindy Truong } 



\section{Operation}

Consider the operation
\[
C := A^TA +C 
\]
where $ A $ is a $ m \times m $ lower triangular matrix and $ C $ is a $ m \times m $ matrix.
This is a special case of triangular 
matrix-matrix multiplication, 
with a matrix 
being multiplied by its transpose on the left.
We will refer to this operation
as {\sc Syrk\_lt} where the {\sc lt} indicates the two matrices are
\underline{t}ransposed and the matrix being updated is stored in the 
\underline{l}ower triangular part.


\section{Precondition and postcondition}

In the precondition 
\[
C = \widehat C
\]
$ \widehat C $ denotes the original contents of $ C $.
This allows us to express the state upon completion, the postcondition, as
\[
C = A^TA + \widehat C.
\]
It is implicitly assumed that $ C $ is a symmetric matrix.
\section{Partitioned Matrix Expressions and loop invariants}

There is one PME for this operation.

\subsection{PME}

To derive the PME, partition
\[
C \rightarrow
\left(\begin{array}{c I c}
C_{TL} & C_{BL}^T \\ \whline
C_{BL} & C_{BR}
\end{array}\right)
,
	\quad \mbox{and} \quad
A \rightarrow \left(\begin{array}{c I c}
A_L & A_R 
\end{array}\right)
\]
Substituting these into the postcondition
yields
\[
\left(\begin{array}{c I c}
C_{TL} & C_{BL}^T \\ \whline
C_{BL} & C_{BR}
\end{array}\right)
=
\left(\begin{array}{c}
A_L^T \\ \whline
A_R^T
\end{array}\right)
\left(\begin{array}{c I c}
A_L & A_R 
\end{array}\right)
+
\left(\begin{array}{c I c}
\widehat C_{TL} &  C_{BL}^T \\ \whline
\widehat C_{BL} & \widehat C_{BR}
\end{array}\right)
\]
or, equivalently,
\[
\left(\begin{array}{c I c}
C_{TL} & C_{BL}^T \\ \whline
C_{BL} & C_{BR}
\end{array}\right)
=
\left(\begin{array}{c I c}
A_L^TA_L + \widehat C_{TL} & A_L^TA_R +  C_{BL}^T \\ \whline
A_R^TA_L + \widehat C_{BL} & A_R^TA_R + \widehat C_{BR}
\end{array}\right)
\]
\\
From this, we can choose four loop invariants:
\begin{description}
	\item
	{\bf Invariant 1:}
	$
	\left(\begin{array}{c I c}
	C_{TL} & C_{BL}^T \\ \whline
	C_{BL} & C_{BR}
	\end{array}\right)
	 = 
	 \left(\begin{array}{c I c}
	 A_L^TA_L + \widehat C_{TL} &  C_{BL}^T \\ \whline
	 \widehat C_{BL} & \widehat C_{BR}
	 \end{array}\right)
	$. \\
	(The top left part has been left alone and the rest have been partially computed).
	\item
	{\bf Invariant 2:}
	$
	\left(\begin{array}{c I c}
	C_{TL} & C_{BL}^T \\ \whline
	C_{BL} & C_{BR}
	\end{array}\right)
	= 
	\left(\begin{array}{c I c}
	\widehat C_{TL} &  C_{BL}^T \\ \whline
	\widehat C_{BL} & A_R^TA_R + \widehat C_{BR}
	\end{array}\right)
	$. \\
	(The bottom right part has been left alone and the rest have been partially computed).
	\item
	{\bf Invariant 3:}
	$
	\left(\begin{array}{c I c}
	C_{TL} & C_{BL}^T \\ \whline
	C_{BL} & C_{BR}
	\end{array}\right)
	= 
	\left(\begin{array}{c I c}
	A_L^TA_L + \widehat C_{TL} &  C_{BL}^T \\ \whline
	A_R^TA_L + \widehat C_{BL} & \widehat C_{BR}
	\end{array}\right)
	$. \\
	(The left part has been left alone and the right part has been partially computed).
	\item
	{\bf Invariant 4:}
	$
	\left(\begin{array}{c I c}
	C_{TL} & C_{BL}^T \\ \whline
	C_{BL} & C_{BR}
	\end{array}\right)
	= 
	\left(\begin{array}{c I c}
	\widehat C_{TL} &  C_{BL}^T \\ \whline
	A_R^TA_L + \widehat C_{BL} & A_R^TA_R + \widehat C_{BR}
	\end{array}\right)
	$. \\
	(The bottom part has been left alone and the top part has been partially computed).
\end{description}

\subsection{Notes}

How do I decide to partition the matrices in the postcondition?

\begin{itemize}
	\item
	Pick a matrix (operand), any matrix.  
	\item 
	If that matrix has 
	\begin{itemize}
		\item 
	a triangular structure (in storage), then you want to either partition is into four quadrants, or not at all.  Symmetric matrices and triangular matrices have a triangular structure (in storage).
		\item
	no particular structure, then you partition it vertically (left-right), horizontally (top-bottom), or not at all.
	\end{itemize}
	\item
	Next, partition the other matrices similarly, but conformally (meaning the 
	resulting multiplications with the parts are legal).
\end{itemize}
Take our problem here:  $ C := A^TA + C $.
Start by partitioning $ A $ horizontally:
\[
\left(\begin{array}{c I c}
A_L & A_R 
\end{array}\right)
\]

Then, the transpose of $A$ will give us:
\[
\left(\begin{array}{c}
A_L^T \\ \whline
A_R^T
\end{array}\right)
\]

which is easier to express as a matrix partitioned vertically. The postcondition will now look like:  
\[
C
=
\left(\begin{array}{c}
A_L^T \\ \whline
A_R^T
\end{array}\right)
\left(\begin{array}{c I c}
A_L & A_R 
\end{array}\right)
+
\widehat C
\]
\\
Now, the way partitioned matrix multiplication works, $A$ would yield a matrix partitioned into quadrants:
\[
C = 
\begin{array}[t]{c}
\underbrace{
\left(\begin{array}{c}
A_L^T \\ \whline
A_R^T
\end{array}\right)
\left(\begin{array}{c I c}
A_L & A_R 
\end{array}\right)
	}\\
	\left(\begin{array}{c I c}
	A_L^TA_L & A_L^TA_R \\ \whline
	A_R^TA_L & A_R^TA_R
	\end{array}\right)
	\end{array}
+ \widehat C
\]
So, we need to also partition $ C $ into quadrants:
\[
\left(\begin{array}{c I c}
C_{TL} & C_{BL}^T \\ \whline
C_{BL} & C_{BR}
\end{array}\right)
=
\left(\begin{array}{c I c}
A_L^TA_L & A_L^TA_R \\ \whline
A_R^TA_L & A_R^TA_R
\end{array}\right)
+
\left(\begin{array}{c I c}
\widehat C_{TL} &  C_{BL}^T \\ \whline
\widehat C_{BL} & \widehat C_{BR}
\end{array}\right)
\] 

\section{Deriving all unblocked algorithms}

The below table summarizes all loop invariants, with links to all files related to this operation.

The worksheet and code skeletons were genered using 
 the \href{http://edx-org-utaustinx.s3.amazonaws.com/UT1401x/LAFFPfC/Spark/index.html}{\ding{42} Spark webpage}.
 

\begin{center}
	\begin{tabular}{| c | l I c | l |} \hline
		& Invariant & Derivations &Implementations \\ \whline
		1 & 
		$ 
		\left(\begin{array}{c I c}
		C_{TL} & C_{BL}^T \\ \whline
		C_{BL} & C_{BR}
		\end{array}\right)
		= 
		\left(\begin{array}{c I c}
		A_L^TA_L + \widehat C_{TL} &  C_{BL}^T \\ \whline
		\widehat C_{BL} & \widehat C_{BR}
		\end{array}\right)
		$
		&
		\href{syrk_lt/Derivations/syrk_lt_unb_var1.pdf}
		{PDF}
		&
		\begin{minipage}{0.3\textwidth}
		\href{syrk_lt/flameatlab/syrk_lt_unb_var1.mlx}
		{syrk\_lt\_unb\_var1.mlx}\\
		\href{syrk_lt/FLAMEC/syrk_lt_unb_var1/syrk_lt_unb_var1.c}
		{syrk\_lt\_unb\_var1.c}
	    \end{minipage}
	    \\ \hline
	    2 & 
	    $
	    \left(\begin{array}{c I c}
	    C_{TL} & C_{BL}^T \\ \whline
	    C_{BL} & C_{BR} 
	    \end{array}\right)
	    = 
	    \left(\begin{array}{c I c}
	    \widehat C_{TL} &  C_{BL}^T \\ \whline
	    \widehat C_{BL} & A_R^TA_R + \widehat C_{BR}
	    \end{array}\right)
	    $
	    &
	    \href{syrk_lt/Derivations/syrk_lt_unb_var2.pdf}
	    {PDF}
	    &
	    \begin{minipage}{0.3\textwidth}
	    	\href{syrk_lt/flameatlab/syrk_lt_unb_var2.mlx}
	    	{syrk\_lt\_unb\_var2.mlx}\\
	    	\href{syrk_tl/FLAMEC/syrk_lt_unb_var2/syrk_lt_unb_var2.c}
	    	{syrk\_lt\_unb\_var2.c}
	    \end{minipage}
	    \\ \hline
	    3 & 
	    $
	    \left(\begin{array}{c I c}
	    C_{TL} & C_{BL}^T \\ \whline
	    C_{BL} & C_{BR}
	    \end{array}\right)
	    = 
	    \left(\begin{array}{c I c}
	    A_L^TA_L + \widehat C_{TL} &  C_{BL}^T \\ \whline
	    A_R^TA_L + \widehat C_{BL} & \widehat C_{BR}
	    \end{array}\right)
	    $
	    &
	    \href{syrk_lt/Derivations/syrkt_lt_unb_var3.pdf}
	    {PDF}
	    &
	    \begin{minipage}{0.3\textwidth}
	    	\href{syrk_lt/flameatlab/syrk_lt_unb_var3.mlx}
	    	{syrk\_lt\_unb\_var3.mlx}\\
	    	\href{syrk_lt/FLAMEC/syrk_lt_unb_var3/syrk_lt_unb_var3.c}
	    	{syrk\_lt\_unb\_var3.c}
	    \end{minipage}
	    \\ \hline
	    4 & 
	    $
	    \left(\begin{array}{c I c}
	    C_{TL} & C_{BL}^T \\ \whline
	    C_{BL} & C_{BR}
	    \end{array}\right)
	    = 
	    \left(\begin{array}{c I c}
	    \widehat C_{TL} &  C_{BL}^T \\ \whline
	    A_R^TA_L + \widehat C_{BL} & A_R^TA_R + \widehat C_{BR}
	    \end{array}\right)
	    $
	    &
	    \href{syrk_lt/Derivations/syrk_lt_unb_var4.pdf}
	    {PDF}
	    &
	    \begin{minipage}{0.3\textwidth}
	    	\href{syrk_lt/flameatlab/syrk_lt_unb_var4.mlx}
	    	{syrk\_lt\_unb\_var4.mlx}\\
	    	\href{syrk_lt/FLAMEC/syrk_lt_unb_var4/syrk_lt_unb_var4.c}
	    	{syrk\_lt\_unb\_var4.c}
	    \end{minipage}
	    \\ \hline
	\end{tabular}
\end{center}